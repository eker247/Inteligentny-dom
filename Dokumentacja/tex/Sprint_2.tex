\section{Sprint 2}
\begin{itemize}
	\item Data rozpoczęcia: 14.11.2017.
	\item Data zakończenia: 28.11.2017.
	\item Scrum Master: Drzyzga Sławomir.
	\item Product Owner: Homoncik Tomasz.
	\item Development Team: Drozd Daniel, Drzyzga Sławomir, Fornagiel Krzysztof, Homoncik Tomasz.
\end{itemize}
\subsection{Cel} 


Aby zapobiec uzyskaniu dostępu do sterowania inteligentnym domem osobom niepowołanym, należy zalogować się do systemu przez wpisanie nazwy użytkownika i hasła. W celu realizacji następnych wymagań użytkowników opisanych w historiach 1, 2 i 6 należy zamontować czujniki temperatury, światła, ruchu, przekaźniki i serwo mechanizmy. Po ukończeniu sprintu powinno być możliwe sterowanie ogrzewaniem ze strony www oraz automatyczne zapalanie światła w chwili zauważenia ruchu przez czujnik PIR. 



\subsection{Sprint Planning/Backlog}

\paragraph{Tytuł zadania.} Zabezpieczenie strony www.
\begin{itemize}
	\item Estymata: szacowana czasochłonność: ,,M''.
\end{itemize}

\paragraph{Tytuł zadania.} Montaż czujnika światła i ruchu.
\begin{itemize}
	\item Estymata: szacowana czasochłonność: ,,S''.
\end{itemize}

\paragraph{Tytuł zadania.} Sterowanie światłem zewnętrznym.
\begin{itemize}
	\item Estymata: szacowana czasochłonność: ,,S''.
\end{itemize}

\paragraph{Tytuł zadania.} Czujnik temperatury i serwo mechanizm.
\begin{itemize}
	\item Estymata: szacowana czasochłonność: ,,S''.
\end{itemize}

\paragraph{Tytuł zadania.} Program sterujący ogrzewaniem.
\begin{itemize}
	\item Estymata: szacowana czasochłonność: ,,S''.
\end{itemize}

% =============================================================	

\subsection{Realizacja}

\paragraph{Tytuł zadania.} Zabezpieczenie strony www.
\subparagraph{Wykonawca.} Fornagiel Krzysztof.
\subparagraph{Realizacja.} Do interfejsu WWW została dodana funkcjonalność logowania. Bez autoryzacji użytkownik nie ma możliwości sterowania ani sprawdzania stanu przekaźników. Zadanie wykonane zgodnie z założonym czasem.

\paragraph{Tytuł zadania.} Montaż czujnika światła i ruchu.
\subparagraph{Wykonawca.} Drozd Daniel.
\subparagraph{Realizacja.} Sprawozdanie z realizacji zadania (w tym ocena zgodności z estymatą). Kod programu (środowisko \texttt{verbatim}): \begin{verbatim}
for (i=1; i<10; i++)
...
\end{verbatim}.

\paragraph{Tytuł zadania.} Sterowanie światłem zewnętrznym (napisanie programu).
\subparagraph{Wykonawca.} Fornagiel Krzysztof, Homoncik Tomasz.
\subparagraph{Realizacja.} Sprawozdanie z realizacji zadania (w tym ocena zgodności z estymatą). Kod programu (środowisko \texttt{verbatim}): \begin{verbatim}
for (i=1; i<10; i++)
...
\end{verbatim}.

\paragraph{Tytuł zadania.} Czujnik temperatury i serwo mechanizm (montaż).
\subparagraph{Wykonawca.} Drozd Daniel.
\subparagraph{Realizacja.} Sprawozdanie z realizacji zadania (w tym ocena zgodności z estymatą). Kod programu (środowisko \texttt{verbatim}): \begin{verbatim}
for (i=1; i<10; i++)
...
\end{verbatim}.

\paragraph{Tytuł zadania.} Program sterujący ogrzewaniem.
\subparagraph{Wykonawca.} Fornagiel Krzysztof, Drzyzga Sławomir.
\subparagraph{Realizacja.} Sprawozdanie z realizacji zadania (w tym ocena zgodności z estymatą). Kod programu (środowisko \texttt{verbatim}): \begin{verbatim}
for (i=1; i<10; i++)
...
\end{verbatim}.

% =============================================================		

\subsection{Sprint Review/Demo}
Sprawozdanie z przeglądu Sprint'u -- czy założony cel (przyrost) został osiągnięty oraz czy wszystkie zaplanowane Backlog Item'y zostały zrealizowane? Demostracja przyrostu produktu.
