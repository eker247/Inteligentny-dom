\section{Sprint 2}
\begin{itemize}
	\item Data rozpoczęcia: 14.11.2017.
	\item Data zakończenia: 28.11.2017.
	\item Scrum Master: Drzyzga Sławomir.
	\item Product Owner: Homoncik Tomasz.
	\item Development Team: Drozd Daniel, Drzyzga Sławomir, Fornagiel Krzysztof, Homoncik Tomasz.
\end{itemize}
\subsection{Cel} 


Aby zapobiec uzyskaniu dostępu do sterowania inteligentnym domem osobom niepowołanym, należy zalogować się do systemu przez wpisanie nazwy użytkownika i~hasła. W celu realizacji następnych wymagań użytkowników opisanych w~historiach 1, 2 i~6 należy zamontować czujniki temperatury, światła, ruchu, przekaźniki i~serwo mechanizmy. Po ukończeniu sprintu powinno być możliwe sterowanie ogrzewaniem ze strony WWW oraz automatyczne zapalanie światła w~chwili zauważenia ruchu przez czujnik PIR. 



\subsection{Sprint Planning/Backlog}

\paragraph{Tytuł zadania.} Zabezpieczenie strony WWW.
\begin{itemize}
	\item Estymata: szacowana czasochłonność: ,,M''.
\end{itemize}

\paragraph{Tytuł zadania.} Montaż czujnika światła i~ruchu.
\begin{itemize}
	\item Estymata: szacowana czasochłonność: ,,S''.
\end{itemize}

\paragraph{Tytuł zadania.} Sterowanie światłem zewnętrznym.
\begin{itemize}
	\item Estymata: szacowana czasochłonność: ,,S''.
\end{itemize}

\paragraph{Tytuł zadania.} Czujnik temperatury i~serwo mechanizm.
\begin{itemize}
	\item Estymata: szacowana czasochłonność: ,,S''.
\end{itemize}

\paragraph{Tytuł zadania.} Program sterujący ogrzewaniem.
\begin{itemize}
	\item Estymata: szacowana czasochłonność: ,,S''.
\end{itemize}

% =============================================================	

\subsection{Realizacja}

\paragraph{Tytuł zadania.} Zabezpieczenie strony WWW.
\subparagraph{Wykonawca.} Fornagiel Krzysztof.
\subparagraph{Realizacja.} Do interfejsu WWW została dodana funkcjonalność logowania. Bez autoryzacji użytkownik nie ma możliwości sterowania ani sprawdzania stanu przekaźników. Zadanie wykonane zgodnie z~założonym czasem.

\paragraph{Tytuł zadania.} Montaż czujnika światła i~ruchu.
\subparagraph{Wykonawca.} Drozd Daniel.
\subparagraph{Realizacja.} 	Połączenia PIR 

\begin{itemize}
	\item +3.3 V
	\item GND
	\item Sygnał - GPIO 18
\end{itemize}
Połączenia fotorezystora :
\begin{itemize}
	\item +3.3 V
	\item GND
	\item Sygnał - GPIO 23
\end{itemize}

\paragraph{Tytuł zadania.} Sterowanie światłem zewnętrznym (napisanie programu).
\subparagraph{Wykonawca.} Fornagiel Krzysztof, Homoncik Tomasz.
\subparagraph{Realizacja.} Program sterujący oświetleniem zewnętrznym uruchomiony jest niezależnie od pozostałej części systemu. Światło w~okół posesji zapalane jest, kiedy na zewnątrz jest ciemno (dane z~czujnika światła) oraz czujnik ruchu wykryje ruch.

\paragraph{Tytuł zadania.} Czujnik temperatury i~serwo mechanizm (montaż).
\subparagraph{Wykonawca.} Drozd Daniel.
\subparagraph{Realizacja.} Połączenie serwomechanizmu oraz czujnika temperatury.\\
Serwomechanizm
\begin{itemize}
	\item + 5 V
	\item GND
	\item Sygnał - GPIO 12
\end{itemize}
Czujnik temperatury
\begin{itemize}
	\item +3.3 V
	\item GND
	\item Sygnał - GPIO 4
\end{itemize}


\paragraph{Tytuł zadania.} Program sterujący ogrzewaniem.
\subparagraph{Wykonawca.} Drzyzga Sławomir, Fornagiel Krzysztof, Homoncik Tomasz.
\subparagraph{Realizacja.} Skrypt odczytujący dane z~czujnika temperatury. Jako argumenty przyjmuje numery pinów z~podłączonymi czujnikami DHT11. Wartością zwracaną jest string z~temperaturą pomieszczeń w~stopniach Celsiusza z~jednym miejscem po przecinku, oddzielone spacjami. 
\break
Skrypt sterujący serwo mechanizmem. Pierwszym argumentem skryptu jest nr pinu, pod który podpięty jest serwo mechanizm. Drugim parametrem jest wartość procentowa od 0 do 100, oznaczająca stopień obrócenia osi mechanizmu. W ten sposób serwo jest podpięty pod zawór kaloryfera jest w~stanie zmniejszać i~zwiększać szybkość przepływu ciepłej wody.
\break
Do sterowania systemem napisany został skrypt w~PHP, uruchomiony niezależnie od serwera. Odczytuje on z~pliku dane o~liczbie pomieszczeń oraz numery pinów dla czujników temperatury i~serwo mechanizmów.
Do strony trzeba było dodać formularz do wprowadzania temperatury dla każdego pomieszczenia.
\subparagraph{}
Skrypt odczytujący temperaturę - temp\_loop. \\
Należy wpisać polecenie sudo ./temp\_loop \& w~celu uruchomienia programu w~tle
\begin{verbatim}
while [ true ]; do 
	echo $(python temperature.py 4 4 4 4 4) > temperature.txt; 
	sleep 15; 
done;
\end{verbatim}


% =============================================================		

\subsection{Sprint Review/Demo}
	Dwutygodniowy sprint zakończony powodzeniem. Klient zaakceptował projekt, można przejść do następnych sprintów. Wszystkie założenia zostały zrealizowane w~terminie. W dniu prezentacji pojawiły się problemy z~zakłóceniami serwomechanizmu, wyregulowano serwomechanizm i~poprawiono skrypt.