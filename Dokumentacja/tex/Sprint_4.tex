\section{Sprint 4}
\begin{itemize}
	\item Data rozpoczęcia:19.12.2017.
	\item Data zakończenia: 16.01.2018.
	\item Scrum Master: Drozd Daniel.
	\item Product Owner: Fornagiel Krzysztof.
	\item Development Team: Drozd Daniel, Drzyzga Sławomir, Fornagiel Krzysztof, Homoncik Tomasz.
\end{itemize}
\subsection{Cel} 

Głównym celem sprintu jest zapewnienie prawidłowego i bezawaryjnego działania inteligentnego domu. Należy skupić się na testowaniu wszystkich elementów systemu i usuwaniu błędów. Jeśli wystarczy czasu należy wdrożyć system zabezpieczający łazienkę przed zalaniem (historia nr 5). Czujniki wykrywające zbyt wysoki poziom wody w wannie lub zlewie będą wysyłać sygnał do serwo sterującego zaworem wody. Aby możliwe było odkręcenie wody po wystąpieniu i zlikwidowaniu skutków zdarzenia, należy zrobić możliwość wysłania sygnału odkręcającego zawór (z poziomu WWW).


\subsection{Sprint Planning/Backlog}

\paragraph{Tytuł zadania.} Testowanie napisanych programów i interfejsu WWW.
\begin{itemize}
	\item Estymata: szacowana czasochłonność: ,,L''.
\end{itemize}

\paragraph{Tytuł zadania.} Montaż czujników zapobiegających przelaniu wody oraz serwo (opcjonalnie).
\begin{itemize}
	\item Estymata: szacowana czasochłonność: ,,S''.
\end{itemize}

\paragraph{Tytuł zadania.} Program zabezpieczający łazienkę przed zalaniem (opcjonalnie).
\begin{itemize}
	\item Estymata: szacowana czasochłonność: ,,S''.
\end{itemize}

% =============================================================	

\subsection{Realizacja}

\paragraph{Tytuł zadania.} Testowanie napisanych programów i interfejsu WWW.
\subparagraph{Wykonawca.} Drozd Daniel, Drzyzga Sławomir, Fornagiel Krzysztof, Homoncik Tomasz.
\subparagraph{Realizacja.}
Wygląda na to, że napisane dotychczas moduły działają prawidłowo. Przez okres 2 tygodni system działał bez zarzutów i nie wykazywał żadnych błędów.

\paragraph{Tytuł zadania.} Montaż czujników zapobiegających przelaniu wody oraz serwo.
\subparagraph{Wykonawca.} Drozd Daniel.
\subparagraph{Realizacja.} Sprawozdanie z realizacji zadania (w tym ocena zgodności z estymatą). Kod programu (środowisko \texttt{verbatim}): \begin{verbatim}
for (i=1; i<10; i++)
...
\end{verbatim}.

\paragraph{Tytuł zadania.} Program zabezpieczający łazienkę przed zalaniem.
\subparagraph{Wykonawca.} Drzyzga Sławomir, Homoncik Tomasz.
\subparagraph{Realizacja.} 
W chwili, kiedy czujnik zalania wysyła sygnał, skrypt uruchamia serwo mechanizm zamykający zawór wody oraz zapisuje informację o tym w pliku. Należało wykonać dodatkowe zmiany w interfejsie WWW jednak całość prac zmieściła się w szacowanym czasie.


% =============================================================		

\subsection{Sprint Review/Demo}
Sprawozdanie z przeglądu Sprint'u -- czy założony cel (przyrost) został osiągnięty oraz czy wszystkie zaplanowane Backlog Item'y zostały zrealizowane? Demonstracja przyrostu produktu.
