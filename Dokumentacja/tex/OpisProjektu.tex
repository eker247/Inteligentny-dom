	\newpage
	
	\section{Opis projektu}
	
	\subsection{Członkowie zespołu}
	
	\begin{enumerate}
		\item Fornagiel Krzysztof (kierownik projektu).
		\item Drozd Daniel.
		\item Drzyzga Sławomir.
		\item Homoncik Tomasz.
	\end{enumerate}
	
	\subsection{Cel projektu (produkt)}Imię i nazwisko
	
	<<Krótko opisać, jaki jest cel realizowanego projektu (określić uzyskiwany produkt)>>.
	
	\subsection{Potencjalny odbiorca produktu (klient)}
	
	<<Określić potencjalnego klienta (wraz z uzasadnieniem)>>.
	
	\subsection{Metodyka}
	
	Projekt będzie realizowany przy użyciu (zaadaptowanej do istniejących warunków) metodyki {\em Scrum}. 
	
	\section{Wymagania użytkownika}
	<<Przedstawić listę wymagań użytkownika w postaci ,,historyjek'' (User stories). Każda historyjka powinna opisywać jedną cechę systemu. Struktura: As a [type of user], I want [to perform some task] so that I can [achieve some goal/benefit/value] (zob. np. \cite{us}).>>
	
	\subsection{User story 1}
	 Zimna noc. Grażyna budzi się chcąc podkręcić ogrzewanie. Niestety, aby to uczynić musi wstać i podejść do grzejnika. Grażyna marzy o tym żeby zrobić to zdalnie przez jej telefon.
	
	\subsection{User story 2}
	Grażyna jest w pracy do późnych godzin wieczornych, myśli o tym, że za niedługo wróci do zimnego domu jednak przypomina sobie, że może podkręcić ogrzewanie korzystając ze swojego laptopa.
	
	\subsection*{<<Tutaj dodawać kolejne historyjki>>}
	
	\section{Harmonogram}
	
	\subsection{Rejestr zadań (Product Backlog)}
	
	\begin{itemize}
		\item Data rozpoczęcia: <<data>>.
		\item  Data zakończenia: <<data>>.
	\end{itemize}