	\newpage
	
	\section{Opis projektu}
	
	\subsection{Członkowie zespołu}
	
	\begin{enumerate}
		\item Fornagiel Krzysztof (kierownik projektu).
		\item Drozd Daniel.
		\item Drzyzga Sławomir.
		\item Homoncik Tomasz.
	\end{enumerate}
	
	\subsection{Cel projektu (produkt)}
	Celem projektu jest wdrożenie systemu sterującego oświetleniem, ogrzewaniem i monitorującego dom za pomocą strony internetowej.
	%<<Krótko opisać, jaki jest cel realizowanego projektu (określić uzyskiwany produkt)>>.
	
	\subsection{Potencjalny odbiorca produktu (klient)}
	Osoby ceniące wygodę i bezpieczeństwo domu, podróżujący i chcący monitorować stan mieszkania.
	
%	<<Określić potencjalnego klienta (wraz z uzasadnieniem)>>.
	
	\subsection{Metodyka}
	
	Projekt będzie realizowany przy użyciu (zaadaptowanej do istniejących warunków) metodyki {\em Scrum}. 
	
	\section{Wymagania użytkownika}
%	<<Przedstawić listę wymagań użytkownika w postaci ,,historyjek'' (User stories). Każda historyjka powinna opisywać jedną cechę systemu. Struktura: As a [type of user], I want [to perform some task] so that I can [achieve some goal/benefit/value] (zob. np. \cite{us}).>>
	
	\subsection{Story 1}
	 Jako użytkownik inteligentnego domu chcę mieć możliwość podkręcenia sobie ogrzewania będąc w łóżku za pomocą telefonu bądź laptopa.
	
	\subsection{Story 2}
	Jako użytkownik inteligentnego domu chcę mieć możliwość w każdym momencie zorientowania się jaka jest aktualna temperatura w pokoju, w którym się znajduję.
	
	\subsection{Story 3}
	Jako użytkownik inteligentnego domu chcę móc zamykać i otwierać rolety w każdej chwili, żeby sąsiedzi nie widzieli co robię.
	
	\subsection{Story 4} 
	Jako użytkownik inteligentego domu chcę, aby rolety zasłaniały okno w momencie kiedy czujnik wykryje, że świeci w nie słońce,żeby nie przeszkadzało mi w oglądaniu telewizji.
	
	\subsection{Story 5}
	Jako użytkownik inteligentnego domu chcę wiedzieć czy w łazience nie przelewa się woda z pralki lub wanny, żeby uniknąć zalania. 
	
	\subsection{Story 6}
	Jako użytkownik inteligentnego domu chcę aby oświetlenie wokół posesji, automatycznie się włączało gdy czujnik wykryje ruch oraz po zmroku.
	
	\subsection{Story 7}
	Jako użytkownik inteligentnego domu chcę mieć możliwość zapalenia i gaszenia światła z telefonu, abym nie musiał podchodzić do kontaktu.
	
	\subsection{Story 8}
	Jako użytkownik inteligentnego domu chcę mieć możliwość wyłączenia/włączenia poszczególnych gniazdek w instalacji domowej, żeby płacić mniejsze rachunki za prąd.
	
	\subsection{Story 9}
	Jako użytkownik inteligentnego domu chcę mieć możliwość sterowania kolorowymi ledami w pokoju, żeby zrobić nastrój.
	
	%\subsection*{<<Tutaj dodawać kolejne historyjki>>}
	
	\section{Harmonogram}
	
	\subsection{Rejestr zadań (Product Backlog)}
	
	\begin{itemize}
		\item Data rozpoczęcia: 24.10.2017.
		\item  Data zakończenia: 09.01.2018.
	\end{itemize}
\subsection{Sprint 1}

\begin{itemize}
	\item Data rozpoczęcia: <<data>>.
	\item Data zakończenia: <<data>>.
	\item Scrum Master: <<imię i nazwisko>>.
	\item Product Owner: <<imię i nazwisko>>.
	\item Development Team: <<lista developerów>>.
\end{itemize}

\subsection{Sprint 2}

\begin{itemize}
	\item Data rozpoczęcia: <<data>>.
	\item  Data zakończenia: <<data>>.
	\item Scrum Master: <<imię i nazwisko>>.
	\item Product Owner: <<imię i nazwisko>>.
	\item Development Team: <<lista developerów>>.
\end{itemize}