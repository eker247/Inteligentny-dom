	\subsection{Sprint 1}
	
	\begin{itemize}
		\item Data rozpoczęcia: <<data>>.
		\item Data zakończenia: <<data>>.
		\item Scrum Master: <<imię i nazwisko>>.
		\item Product Owner: <<imię i nazwisko>>.
		\item Development Team: <<lista developerów>>.
	\end{itemize}
	
	\section{Sprint 1}
	\subsection{Cel} <<Określić, w jakim celu tworzony jest przyrost produktu>>.
	\subsection{Sprint Planning/Backlog}
	
	\paragraph{Tytuł zadania.} <<Tytuł>>.
	\begin{itemize}
		\item Estymata: <<szacowana czasochłonność (w ,,koszulkach'')>>.
	\end{itemize}
	
	\paragraph{Tytuł zadania.} <<Tytuł>>.
	\begin{itemize}
		\item Estymata: <<szacowana czasochłonność (w ,,koszulkach'')>>.
	\end{itemize}
	
	\paragraph{<<Tutaj dodawać kolejne zadania>>}
	
	\subsection{Realizacja}
	
	\paragraph{Tytuł zadania.} <<Tytuł>>.
	\subparagraph{Wykonawca.} <<Wykonawca>>.
	\subparagraph{Realizacja.} <<Sprawozdanie z realizacji zadania (w tym ocena zgodności z estymatą). Kod programu (środowisko \texttt{verbatim}): \begin{verbatim}
	for (i=1; i<10; i++)
	...
	\end{verbatim}>>.
	
	\paragraph{Tytuł zadania.} <<Tytuł>>.
	\subparagraph{Wykonawca.} <<Wykonawca>>.
	\subparagraph{Realizacja.} <<Sprawozdanie z realizacji zadania (w tym ocena zgodności z estymatą). Kod programu (środowisko \texttt{verbatim}): \begin{verbatim}
	for (i=1; i<10; i++)
	...
	\end{verbatim}>>.
	
	\paragraph{<<Tutaj dodawać kolejne zadania>>}
	
	
	\subsection{Sprint Review/Demo}
	<<Sprawozdanie z przeglądu Sprint'u -- czy założony cel (przyrost) został osiągnięty oraz czy wszystkie zaplanowane Backlog Item'y zostały zrealizowane? Demostracja przyrostu produktu>>.
	