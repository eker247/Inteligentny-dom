	\subsection{Sprint 1}
	
	\begin{itemize}
		\item Data rozpoczęcia: 31.10.2017.
		\item Data zakończenia: 14.11.2017.
		\item Scrum Master: Homoncik Tomasz.
		\item Product Owner: Drzyzga Sławomir.
		\item Development Team: Drozd Daniel, Fornagiel Krzysztof.
	\end{itemize}
	
	\section{Sprint 1}
	\subsection{Cel} 

	
	W celu zrealizowania zadań z historii 7 i 8 musimy wykonać stronę internetową do sterowania z poziomu przeglądarki internetowej. Dodatkowo należy wykonać połączenie 3 czujników(Pir, przekaźniki, fotorezystor) i napisać odpowiednie skrypty. Instalacja VPN.
	
	
	
	\subsection{Sprint Planning/Backlog}
	
	\paragraph{Tytuł zadania.} Tytuł.
	\begin{itemize}
		\item Estymata: szacowana czasochłonność (w ,,koszulkach'').
	\end{itemize}
	
	\paragraph{Tytuł zadania.} Tytuł.
	\begin{itemize}
		\item Estymata: szacowana czasochłonność (w ,,koszulkach'').
	\end{itemize}
	
	\paragraph{Tutaj dodawać kolejne zadania}
	
	\subsection{Realizacja}
	
	\paragraph{Tytuł zadania.} Tytuł.
	\subparagraph{Wykonawca.} Wykonawca.
	\subparagraph{Realizacja.} Sprawozdanie z realizacji zadania (w tym ocena zgodności z estymatą). Kod programu (środowisko \texttt{verbatim}): \begin{verbatim}
	for (i=1; i<10; i++)
	...
	\end{verbatim}.
	
	\paragraph{Tytuł zadania.} Tytuł.
	\subparagraph{Wykonawca.} Wykonawca.
	\subparagraph{Realizacja.} Sprawozdanie z realizacji zadania (w tym ocena zgodności z estymatą). Kod programu (środowisko \texttt{verbatim}): \begin{verbatim}
	for (i=1; i<10; i++)
	...
	\end{verbatim}.
	
	\paragraph{Tutaj dodawać kolejne zadania}
	
	
	\subsection{Sprint Review/Demo}
	Sprawozdanie z przeglądu Sprint'u -- czy założony cel (przyrost) został osiągnięty oraz czy wszystkie zaplanowane Backlog Item'y zostały zrealizowane? Demostracja przyrostu produktu.
	