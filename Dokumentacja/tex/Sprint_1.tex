\section{Sprint 1}
	\begin{itemize}
		\item Data rozpoczęcia: 31.10.2017.
		\item Data zakończenia: 14.11.2017.
		\item Scrum Master: Homoncik Tomasz.
		\item Product Owner: Drzyzga Sławomir.
		\item Development Team: Drozd Daniel, Drzyzga Sławomir, Fornagiel Krzysztof, Homoncik Tomasz.
	\end{itemize}
\subsection{Cel} 

	
	W celu zrealizowania zadań z historii 7 i 8 musimy wykonać szereg dodatkowych czynności. Aby było możliwe zdalne sterowanie przez stronę www należy skonfigurować router, podłączyć komputer, zainstalować serwer www, interpreter PHP, bazę danych, aplikację ddns, firewall, napisać pierwszą wersję strony internetowej z możliwością sterowania światłem i przekaźnikami. Dodatkowo należy wykonać połączenie 3 czujników (Pir, przekaźniki, fotorezystor) i napisać odpowiednie skrypty.
	
	
	
\subsection{Sprint Planning/Backlog}
	
	\paragraph{Tytuł zadania.} Konfiguracja infrastruktury sieciowej.
	\begin{itemize}
		\item Estymata: szacowana czasochłonność: ,,S''.
	\end{itemize}
	
	\paragraph{Tytuł zadania.} Przygotowanie serwera www.
	\begin{itemize}
		\item Estymata: szacowana czasochłonność: ,,S''.
	\end{itemize}
	
	\paragraph{Tytuł zadania.} Zabezpieczenie przed niepożądanym ruchem z sieci.
	\begin{itemize}
		\item Estymata: szacowana czasochłonność: ,,S''.
	\end{itemize}
	
	\paragraph{Tytuł zadania.} Stworzenie interfejsu.
	\begin{itemize}
		\item Estymata: szacowana czasochłonność: ,,M''.
	\end{itemize}
	
	\paragraph{Tytuł zadania.} Montaż przekaźników do sterowania światłem i gniazdami.
	\begin{itemize}
		\item Estymata: szacowana czasochłonność: ,,S''.
	\end{itemize}
	
	\paragraph{Tytuł zadania.} Napisanie programu sterującego światłem i gniazdami.
	\begin{itemize}
		\item Estymata: szacowana czasochłonność: ,,S''.
	\end{itemize}

% =============================================================	

	\subsection{Realizacja}
	
	\paragraph{Tytuł zadania.} Konfiguracja infrastruktury sieciowej.
	\subparagraph{Wykonawca.} Drozd Daniel.
	\subparagraph{Realizacja.} Przygotowanie serwera - raspberry pi, instalacja systemu Debian w wersji Raspbian. Przygotowanie serwera www, instalacja : Apache2 + PHP + MySQL + phpMyAdmin.
							   Stworzenie połączeń dla czujnika PIR, fotorezystora oraz przekaźnika. Stworzenie połączenia dynamicznego DNS, przekierowanie sieci na adres gerwant222.ddns.net
							   Przekierowanie odpowiednich portów routera do sieci zewnętrze.
							   \\Połączenia PIR :
							   \begin{itemize}
							   	\item +3.3 V
							   	\item GND
							   	\item Sygnał - GPIO X
							   \end{itemize}
						   	   Połączenia fotorezystora :
						   	   \begin{itemize}
						   	   	\item +3.3 V
						   	   	\item GND
						   	   	\item Sygnał - GPIO X
						   	   \end{itemize}
						   	   Połączenia przekaźnika : 
						   	   \begin{itemize}
						   	   	\item +3.3 V
						   	   	\item GND
						   	   	\item Sygnał - GPIO X
						   	   \end{itemize}
						   	   
							   
	
	\paragraph{Tytuł zadania.} Przygotowanie serwera www.
	\subparagraph{Wykonawca.} Drzyzga Sławomir.
	\subparagraph{Realizacja.} Sprawozdanie z realizacji zadania (w tym ocena zgodności z estymatą). Kod programu (środowisko \texttt{verbatim}): \begin{verbatim}
	for (i=1; i<10; i++)
	...
	\end{verbatim}.
	
	\paragraph{Tytuł zadania.} Zabezpieczenie przed niepożądanym ruchem z sieci.
	\subparagraph{Wykonawca.} Homoncik Tomasz.
	\subparagraph{Realizacja.} Sprawozdanie z realizacji zadania (w tym ocena zgodności z estymatą). Kod programu (środowisko \texttt{verbatim}): \begin{verbatim}
	for (i=1; i<10; i++)
	...
	\end{verbatim}.
	
	\paragraph{Tytuł zadania.} Stworzenie interfejsu www.
	\subparagraph{Wykonawca.} Fornagiel Krzysztof.
	\subparagraph{Realizacja.} Sprawozdanie z realizacji zadania (w tym ocena zgodności z estymatą). Kod programu (środowisko \texttt{verbatim}): \begin{verbatim}
	for (i=1; i<10; i++)
	...
	\end{verbatim}.
	
	\paragraph{Tytuł zadania.} Montaż przekaźników do sterowania światłem i gniazdami.
	\subparagraph{Wykonawca.} Drozd Daniel.
	\subparagraph{Realizacja.} Sprawozdanie z realizacji zadania (w tym ocena zgodności z estymatą). Kod programu (środowisko \texttt{verbatim}): \begin{verbatim}
	for (i=1; i<10; i++)
	...
	\end{verbatim}.
	
	\paragraph{Tytuł zadania.} Napisanie programu sterującego światłem i gniazdami.
	\subparagraph{Wykonawca.} Fornagiel Krzysztof.
	\subparagraph{Realizacja.} Sprawozdanie z realizacji zadania (w tym ocena zgodności z estymatą). Kod programu (środowisko \texttt{verbatim}): \begin{verbatim}
	for (i=1; i<10; i++)
	...
	\end{verbatim}.

% =============================================================		
	
	\subsection{Sprint Review/Demo}
	Sprawozdanie z przeglądu Sprint'u -- czy założony cel (przyrost) został osiągnięty oraz czy wszystkie zaplanowane Backlog Item'y zostały zrealizowane? Demostracja przyrostu produktu.
	