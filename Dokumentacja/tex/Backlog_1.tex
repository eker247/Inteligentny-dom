%	\section{Harmonogram}
%	\subsection{Rejestr zadań (Product Backlog)}
	


\section{Product Backlog}
	\begin{itemize}
		\item Data rozpoczęcia: 24.10.2017.
		\item  Data zakończenia: 31.10.2017.
	\end{itemize}

	\subsection{Backlog Item 1}
	\paragraph{Tytuł zadania:}
	Konfiguracja infrastruktury sieciowej.
	
	\paragraph{Opis zadania:}
	W miejscu podłączenia mikrokomputera do sieci Internet należy skonfigurować router tak, aby kierował ruch sieciowy na urządzenie i aby po wpisaniu do przeglądarki adresu pojawiała się strona naszego projektu. Na mikrokomputerze należy zainstalować aplikację ddns oraz zarejestrować wybrany adres.
	
	\paragraph{Priorytet:}
	5
	
	\paragraph{Definition of Done:}
	Po wpisaniu w dowolnej przeglądarce na dowolnym urządzeniu adresu strony www pokaże się panel sterowania urządzeniami w inteligentnym domu.
	
% =============================================================	

\subsection{Backlog Item 2}
\paragraph{Tytuł zadania:}
Zabezpieczenie przed niepożądanym ruchem z sieci.

\paragraph{Opis zadania:}
Instalacja i konfiguracja firewalla na serwerze. Należy zablokować porty niewykorzystywane do tworzenia, uaktualniania i testowania systemu.

\paragraph{Priorytet:}
5

\paragraph{Definition of Done:}
Z poziomu wiersza poleceń można sprawdzić stan blokowanych portów przez firewall.

% =============================================================	

\subsection{Backlog Item 3}
\paragraph{Tytuł zadania:}
Przygotowanie serwera www.

\paragraph{Opis zadania:}
Instalacja serwera apache, języka PHP i bazy danych.

\paragraph{Priorytet:}
5

\paragraph{Definition of Done:}
Prawidłowo skonfigurowany serwer interpretuje skrypty PHP. Nie wysyła poleceń języka na przeglądarkę klienta. Jest w stanie prawidłowo wyświetlać stronę.

% =============================================================	

	\subsection{Backlog Item 4}
	\paragraph{Tytuł zadania:}
	Stworzenie interfejsu.
	
	\paragraph{Opis zadania:}
	Napisanie strony internetowej w PHP, umożliwiającej sterowanie wszystkimi zainstalowanymi w danej chwili urządzeniami. Strona powinna być uaktualniana w trakcie dołączania kolejnych czujników i urządzeń.
	
	\paragraph{Priorytet:}
	5
	
	\paragraph{Definition of Done:}
	Dzięki stronie www użytkownik może sterować inteligentnym domem. Na stronie znajdują się przyciski włączające i wyłączające światło i przekaźniki.

% =============================================================	

	\subsection{Backlog Item 5}
	\paragraph{Tytuł zadania:}
	Zabezpieczenie strony www.
	
	\paragraph{Opis zadania:}
	Na stronie należy wykonać możliwość logowania aby osoby nieuprawnione nie miały możliwości sterowania naszym systemem.
	
	\paragraph{Priorytet:}
	5
	
	\paragraph{Definition of Done:}
	Bez zalogowania na stronie nie będzie możliwości wysłania żadnych poleceń do serwera.

% =============================================================	
	
	\subsection{Backlog Item 6}
	\paragraph{Tytuł zadania:} 
	Montaż przekaźników do sterowania światłem i gniazdami.
	
	\paragraph{Opis zadania:} 
	Montaż przekaźników do gniazd i lamp. Poprowadzenie przewodu sterującego i podpięcie do mikro-kontrolera.
	
	\paragraph{Priorytet:}
	4
	
	\paragraph{Definition of Done:}
	Z poziomu wiersza poleceń możliwe jest uaktywnienie lub deaktywacja przekaźnika (łączenie obwodu elektrycznego w lampach lub gniazdach).

% =============================================================

	\subsection{Backlog Item 7}
	\paragraph{Tytuł zadania:} 
	Napisanie programu sterującego światłem i gniazdami.
	
	\paragraph{Opis zadania:} 
	W odniesieniu do historii nr 7 i 8 należy napisać program odpowiedzialny za sterowanie przekaźnikami w gniazdkach i lampach.
	
	\paragraph{Priorytet:}
	4
	
	\paragraph{Definition of Done:}
	Możliwe stanie się włączanie i gaszenie światła w ustalonych godzinach lub w danym momencie przez stronę internetową.

% =============================================================

	\subsection{Backlog Item 8}
	\paragraph{Tytuł zadania:}
	Montaż czujnika światła i ruchu.
	
	\paragraph{Opis zadania:} 
	Rozmieszczenie czujników, doprowadzenie przewodów, tak aby sprostać wymaganiom użytkownika z historii 6. 
	
	\paragraph{Priorytet:}
	3
	
	\paragraph{Definition of Done:}
	Z poziomu wiersza poleceń można sprawdzić stan czujników.
	

% =============================================================

	\subsection{Backlog Item 9}
	\paragraph{Tytuł zadania:}
	Sterowanie światłem zewnętrznym.
	
	\paragraph{Opis zadania:} 
	Napisanie programu zapalającego zewnętrzne lampy po zmroku w chwili wykrycia ruchu. 
	
	\paragraph{Priorytet:}
	3
	
	\paragraph{Definition of Done:}
	Światła w okół domu zapalają się po zmroku w chwili pojawienia się ruchu na posesji.


% =============================================================

	\subsection{Backlog Item 10}
	\paragraph{Tytuł zadania:}
	Czujnik temperatury i serwo mechanizm.
	
	\paragraph{Opis zadania:} 
	Użytkownik z historii 1 oczekuje zdalnej możliwości regulacji temperatury. Należy rozmieścić czujniki temperatury w pomieszczeniach oraz zamontować mechanizmy regulujące przepływ ciepłej wody przez kaloryfery. Zamontowane czujniki umożliwią również sprawdzanie temperatury w pomieszczeniach jak w historii nr 2.
	
	\paragraph{Priorytet:}
	3
	
	\paragraph{Definition of Done:} 
	Z poziomu wiersza poleceń możliwe jest odczytanie temperatury z czujników oraz sterowanie mechanizmem regulującym.
	

% =============================================================	

\subsection{Backlog Item 11}
\paragraph{Tytuł zadania:}
Program sterujący ogrzewaniem.

\paragraph{Opis zadania:} 
Napisanie programu odczytującego obecną temperaturę w pomieszczeniu, oczekiwaną temperaturę przez użytkownika oraz odpowiednio regulujący kaloryfery.

\paragraph{Priorytet:}
3

\paragraph{Definition of Done:} 
Użytkownik może wprowadzić żądaną temperaturę w pomieszczeniu na stronie internetowej. Program w przypadku odczytania z czujnika zbyt niskiej temperatury wyśle do serwo sygnał, aby ten umożliwił szybszy przepływ wody. W przypadku osiągnięcia zbyt wysokiej temperatury w pomieszczeniu, kaloryfer powinien zostać zakręcony. 


% =============================================================	
	
	\subsection{Backlog Item 12}
	\paragraph{Tytuł zadania:}
	Sterowanie roletami.
	
	\paragraph{Opis zadania:} 
	Montaż serwo mechanizmów na roletach w pomieszczeniach oraz czujnika zmierzchu wg oczekiwań użytkownika z historii 3. Opcjonalnie można zamontować dodatkowy czujnik światła w okolicy telewizora nawiązując do historii 4.
	
	\paragraph{Priorytet:}
	3
	
	\paragraph{Definition of Done:}
	Możliwość sterowania roletami z poziomu wiersza poleceń.


% =============================================================	

	\subsection{Backlog Item 13}
	\paragraph{Tytuł zadania:}
	Sterowanie roletami ze strony www.
	
	\paragraph{Opis zadania:} 
	Stworzenie programu umożliwiającego zasłanianie i odsłanianie okien.
	
	\paragraph{Priorytet:}
	3
	
	\paragraph{Definition of Done:}
	Z poziomu przeglądarki www możliwe jest automatyczne zasłanianie okien po zmierzchu, odsłanianie okien wraz ze wschodem Słońca lub o określonych godzinach. Sterowanie możliwe jest również na żądanie w danej chwili.


% =============================================================	
	
	\subsection{Backlog Item 14}
	\paragraph{Tytuł zadania:}
	Sterowanie kolorowym światłem.
	
	\paragraph{Opis zadania:}
	Montaż lamp LED umożliwiających uzyskanie w pomieszczeniu światła o wymaganym kolorze, aby użytkownik z historii 9 miał możliwość stworzenia odpowiedniej atmosfery w zależności od potrzeby.
	 
	\paragraph{Priorytet:}
	2
	
	\paragraph{Definition of Done:}
	Lampa LED zmienia kolor sterowana z poziomu wiersza poleceń.


% =============================================================	

	\subsection{Backlog Item 15}
	\paragraph{Tytuł zadania:}
	Program sterujący LED-ami.
	
	\paragraph{Opis zadania:}
	Napisanie programu, dzięki któremu użytkownik przez stronę www może wybrać kolor światła.
	
	\paragraph{Priorytet:}
	2
	
	\paragraph{Definition of Done:}
	Możliwość zapalania, gaszenia oraz wybór koloru światła.


% =============================================================	
	
	\subsection{Backlog Item 16}
	\paragraph{Tytuł zadania:}
	Montaż czujników zapobiegających przelaniu wody oraz serwo (opcjonalnie).
	
	\paragraph{Opis zadania:} 
	Instalacja czujników poziomu wody w pomieszczeniach narażonych na zalanie zgodnie z historią nr 5.
	
	\paragraph{Priorytet:} 
	1
	
	\paragraph{Definition of Done:}
	Z poziomu wiersza poleceń można sprawdzić stan czujnika wysokości wody. Można również zakręcić zawór wody wysyłając sygnał do serwo mechanizmu.

% =============================================================	

	\subsection{Backlog Item 17}
	\paragraph{Tytuł zadania:}
	Program zabezpieczający łazienkę przed zalaniem (opcjonalnie).
	
	\paragraph{Opis zadania:} 
	Napisanie programu, który w przypadku otrzymania sygnału od czujnika wody zakręci główny zawór wody. 
	
	\paragraph{Priorytet:} 
	1
	
	\paragraph{Definition of Done:}
	W chwili, gdy czujnik wody odczyta informację o niebezpiecznej sytuacji, uruchomi serwo mechanizm zakręcający główny zawór wody.

% =============================================================	

\subsection{Backlog Item 18}
\paragraph{Tytuł zadania:}
Testowanie napisanych programów i interfejsu www.

\paragraph{Opis zadania:} 
Sprawdzanie bezawaryjnego działania systemu, szukanie i usuwanie błędów w programach.

\paragraph{Priorytet:} 
1

\paragraph{Definition of Done:}
Bezawaryjne działanie systemu.

% =============================================================	

%   Backlog item zapas	
%	\subsection{Backlog Item 4}
%	\paragraph{Tytuł zadania.} Tytuł.
%	\paragraph{Opis zadania.} 
%	\paragraph{Priorytet.} Priorytet.
%	\paragraph{Definition of Done.} Określić (w języku zrozumiałym dla wszystkich członków zespołu), co oznacza ukończenie danego zadania.	
	
